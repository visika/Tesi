%*******************************************************
% Abstract
%*******************************************************
%\renewcommand{\abstractname}{Abstract}
\pdfbookmark[1]{Sommario}{Sommario}
% \addcontentsline{toc}{chapter}{\tocEntry{Abstract}}
\begingroup
\let\clearpage\relax
\let\cleardoublepage\relax
\let\cleardoublepage\relax

\chapter*{Sommario}
%enuncia il problema
Il campo del quantum machine learning è tra gli ambiti di ricerca più 
promettenti per quanto riguarda il miglioramento dell'analisi dei big data. 
%perché il problema è un problema
I problemi predominanti con cui devono confrontarsi i computer classici sono i 
limiti di memoria e la capacità di elaborazione in tempi appropriati.
% frase sorprendente
Sfruttando le proprietà quantomeccaniche della materia, i computer quantistici 
possono veicolare nuove metodologie d'analisi dati. 
% consguenze della frase sorprendente
A tal fine si esplorano le possibilità offerte dall'implementazione di un algoritmo KNN, 
attraverso il kit di sviluppo software Qiskit e la piattaforma online IBM Q Experience. 

% ML, problemi del ML
% QML come aiuta
% Meno accento sui big data, giusto come conseguenza o in conclusione


\vfill

\begin{otherlanguage}{english}
\pdfbookmark[1]{Abstract}{Abstract}
\chapter*{Abstract}
Quantum Machine Learning is among the most anticipated fields when dealing 
with improvements in big data analysis. 
The main issues classical computers have to confront with are limits in memory 
and enormous computation times. 
Making use of quantum mechanical properties of matter, quantum computers can
spur the development of new methods to improve big data analysis. 
With that in mind, this thesis explores the capabilities of an implementation of a KNN 
algorithm, using the Qiskit software development kit and the online platform IBM Q Experience. 
\end{otherlanguage}

\endgroup

\vfill
