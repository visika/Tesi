%************************************************
\chapter{Introduzione}\label{ch:introduzione}
%************************************************

% L'abilità di riconoscere modelli da parte delle macchine è una cosa molto bella. 
% Per aiutare i computer dei nostri giorni che non ce la fanno più, si stanno ideando 
% varie soluzioni. Qui parlo dei computer quantistici, che sfruttano proprietà 
% della materia quando è molto fredda, che ci permettono di fare cose strabilianti. 

% L'articolo con cui è partita la mia attività di tesi è quello di Tacchino, 
% in cui implementa un percettrone a due qubit, che è capace di riconoscere 
% motivi a quattro bit. Quello su cui mi concentro io è un algoritmo analogo 
% e ugualmente semplice, detto k-Nearest Neighbour, ovvero i k vicini più vicini. 

% Per affrontare l'argomento sono partito dalla tesi di Mark Fingerhuth, che 
% ha implementato un algoritmo ma non è riuscito a farlo girare su un vero 
% computer quantistico, perché non ce n'erano ancora di abbastanza potenti. 
% Adesso ci sono, e ci divertiamo. 

%****************************************

Chiunque abbia un cellulare di nuova generazione è venuto in contatto con il 
campo del machine learning, la disciplina che si pone l'obiettivo di rendere 
i computer capaci di apprendere dall'esperienza. 
Gli esseri umani sono strettamente legati al processo di apprendimento, a 
partire dalla prima infanzia ed in maniera continua durante tutta la loro vita. 
Sebbene tale processo sembri automatico, in realtà è espressione di un 
complesso sistema di feedback, che è codificato dal nostro codice genetico. 
Il \ac{ML} (tradotto \emph{apprendimento automatico}) prevede la scrittura 
del codice di programmazione che permetta 
ai computer di effettuare qualcosa di analogo all'apprendimento. 
\marginpar{Un esempio recente di uso del \ac{ML} è la funzione di gestione adattiva della batteria, 
lanciata in Android 9, che usa il machine learning per predire quali applicazioni 
l'utente userà nelle prossime ore e quali no, in modo da dedicare la carica solo per le 
applicazioni più usate dal proprietario. \cite{android10}}
Tali algoritmi sono il motore dei moderni assistenti vocali e dei sistemi 
che ci suggeriscono il prodotto da acquistare o il video da guardare più affine ai nostri gusti, 
imparando dall'esperienza; ma sono impiegati anche 
nel prevenire le frodi su carta di credito, per filtrare lo spam dalle nostre caselle 
email, nell'individuare e diagnosticare malattie, e tanto ancora \cite{pml}. 

% L'abilità di riconoscere facce familiari, di comprendere il linguaggio parlato e di 
% distinguere tra tipi di piante viene naturale agli esseri umani, sebbene
% % sebbene, benché, quantunque, anche se
% questi processi di riconoscimento di modelli e di classificazione siano intrinsecamente 
% complessi. Il \ac{ML}, una sottodisciplina dell'intelligenza artificiale, 
% si occupa dello sviluppo di algoritmi che eseguono questo tipo di compiti, permettendo 
% in questo modo % così
% ai computer di trovare e riconoscere modelli nei dati e di classificare input 
% % dati da elaborare
% sconosciuti % ignoti
% basandosi sui % in base ai
% precedenti dati di addestramento. % allenamento, ammaestramento
% Tali algoritmi formano il nucleo di p.e. motori di riconoscimento del linguaggio % parola
% umano e di raccomandazione come quelli usati da Amazon. % Fonti

Secondo l'IBM \cite{ibm-big-data}, ogni giorno vengono creati circa $2,5\times10^{18}$ byte di dati: 
questo numero in costante crescita mette chi ha a che fare con i dati di fronte 
alla necessità di dotarsi di algoritmi avanzati che possano fare ordine in questo 
oceano di informazioni. Per trovare modelli e correlazioni su insiemi molto grandi
è necessario un numero di operazioni che richiede molto tempo ed 
energia per essere portate a termine; trovare metodi efficienti per completare 
questi compiti diventa necessario sotto molti punti di vista. 

Diversi lavori hanno evidenziato il ruolo fondamentale rispetto a questo problema 
che può avere l'uso della computazione quantistica, 
oggetto da vari decenni di intensa ricerca teorica, 
e più recentemente anche di ricerca sperimentale \cite{schuld} e tesi di laurea \cite{fingerhuth}. 

Un \ac{CQ} (o \ac{QC}) usa le proprietà peculiari dei sistemi quantomeccanici per manipolare 
ed elaborare l'informazione in modi inaccessibili ai normali computer classici. 
Così come un computer classico manipola bit, un computer quantistico fa uso dei 
cosiddetti bit quantistici (qubit). I bit e i qubit sono entrambi entità binarie, 
il che significa che possono assumere solo i valori 0 o 1. 
Un bit classico non probabilistico\footnote{Per informazioni sui bit classici 
probabilitici si faccia riferimento alla sezione 3.2 di \cite{mannucci}.} 
può assumere solo uno di questi valori alla volta, 
laddove un qubit può trovarsi in una sovrapposizione lineare dei due stati. 
Il fatto di lavorare con una sovrapposizione di stati di un sistema, 
invece che con uno stato definito alla volta, implica che si possono effettuare 
determinate operazioni che coinvolgono più stati contemporaneamente. 
% Questa possibilità è entusiasmante, ma di fronte a nuovi modi di 
% agire, si rende necessario sviluppare nuovi algoritmi e strategie di sviluppo. 

% "Stai pensando troppo classicamente"

Nonostante ciò, uno degli ostacoli principali di questo campo è rappresentato proprio dalla 
caratteristica alla base della meccanica quantistica: quando si ha un qubit 
in una sovrapposizione di stati, non possiamo conoscerne i dettagli intrinseci 
in un determinato istante, a meno di effettuare una misura; se ci provassimo, 
otterremmo un risultato ben definito, facendo collassare la funzione d'onda 
che descrive il qubit, riducendolo in fin dei conti ad un bit con uno stato definito (esattamente 
quello che misuriamo), e perdendo tutte le informazioni quantistiche 
contenute nel qubit. 

% Il tentativo negli ultimi anni è stato quello di costruire macchine 
% che sfruttassero queste proprietà a proprio vantaggio, a patto di obbedire 
% alle regole del gioco. Quando abbiamo un qubit in una sovrapposizione di 
% stati, non possiamo conoscere in quale stato si trovi in un determinato istante; 
% se ci provassimo, misurandolo, otterremmo un risultato ben definito, ma lo 
% avremmo fatto collassare e distrutto tutte le informazioni quantistiche che 
% conteneva. 

Per avere un'idea di come un \ac{CQ} si rapporti con un computer ordinario, 
si possono considerare diversi algoritmi quantistici, che forniscono 
miglioramenti esponenziali nel numero di operazioni se confrontati con le loro controparti classiche; 
l'algoritmo di fattorizzazione in numeri primi di Shor è uno tra i più famosi \cite{shor}. 
% La computazione quantistica ha il potenziale di migliorare la potenza computazionale, 
% accelerare l'elaborazione di grandi dati e probabilmente risolvere alcuni problemi che sono 
% praticamente irrisolvibili su computer classici, p.e. problemi di ottimizzazione 
% computazionalmente dispendiosi come il ben noto problema del commesso viaggiatore con più 
% di 1000 città. % Fonti

% adesso è già avviata la combinazione tra ML e QC

Alla luce di questi fatti, negli ultimi anni c'è stata particolare attenzione nei 
riguardi del nuovo campo del \ac{MLQ} (o \ac{QML}) \cite{Biamonte2017} , che mette insieme il 
\ac{ML} con il quantum computing. 

Più nello specifico, si parla di machine learning migliorato quantisticamente quando 
la computazione quantistica è usata per migliorare algoritmi di \ac{ML} classico. 
Cionondimeno, entrambi i campi sono benefici l'un l'altro, 
poiché il \ac{ML} classico può anche essere usato per migliorare aspetti 
della computazione quantistica \cite{PhysRevLett.116.230504}. 
% Per esempio, Las Heras, Alvarez-Rodriguez, Solano e Sanz % Fonti
% hanno mostrato che un algoritmo genetico classico può essere usato per ridurre 
% l'errore sperimentale nelle porte logiche quantistiche. 
Questa tesi si occuperà solo del campo del machine 
learning migliorato quantisticamente. 

% ***********************************************

\section{Motivazione}

Il \ac{QML} è un campo molto giovane e in fermento. Stiamo assistendo a qualcosa di 
analogo a quella che era la nascita dei computer classici, ovvero lo sviluppo di un 
numero sempre crescente di algoritmi, tecniche ed hardware di base in 
un ambiente in cui le cose consolidate sono ben poche. \marginpar{I campi che si 
prevede otterranno accelerazioni maggiori grazie al quantum computing sono il 
machine learning, la simulazione di sistemi quantomeccanici, i problemi di 
ottimizzazione e l'analisi finanziaria. \cite{what-is-ibm-q}} 
In ogni caso c'è necessità di inventare nuove strategie per risolvere anche i 
problemi più semplici: per avere risultati affidabili spesso si richiede un 
numero considerevole di qubit ed unità di memorizzazione che possano conservare 
informazioni quantistiche, come la \ac{QRAM}. 
Al giorno d'oggi il numero massimo di qubit superconduttivi, da quanto viene 
riferito, è di 50 \cite{50qubit}. 

Negli ultimi anni ci sono state varie implementazioni innovative di algoritmi di \ac{MLQ} \cite{quantum-advantage}: 
seguendo i passi che hanno portato alla nascita del \ac{ML} nel XX secolo, 
Tacchino et al. \cite{tacchino} hanno implementato un percettrone su hardware quantistico;
Schuld et al. \cite{schuld} hanno invece implementato l'algoritmo lazy learner \ac{KNN}. 
Questa tesi parte proprio da quest'ultimo articolo, provando a riprodurne il lavoro e ad 
implementarne una versione multiclasse. 

% Li, Liu, Xu e Du hanno distinto con successo cifre scritte a mano usando un quantum support 
% vector machine su un banco di prova a risonanza magnetica nucleare con quattro qubit. In più, 
% Cai et al. sono stati i primi a dimostrare sperimentalmente il machine learning quantistico su 
% un \ac{CQ} fotonico e hanno mostrato che la distanza tra due vettori e il loro prodotto 
% interno possono effettivamente essere computati quantomeccanicamente. 
% Infine, Ristè et al. hanno risolto un problema di learning parity con cinque qubit 
% superconduttivi e hanno trovato che un vantaggio quantistico può già essere osservato 
% in sistemi senza correzione degli errori. 

% Considerando il distacco tra il numero di algoritmi di \ac{MLQ} e le poche realizzazioni 
% sperimentali, considerando inoltre i recenti sviluppi nell'hardware quantistico, diventa 
% importante scoprire come i problemi di \ac{MLQ} realizzati di \ac{CQ} di piccola dimensione 
% scalino quando implementati su \ac{CQ} con maggiori risorse. 
% Di qui, l'obiettivo di questa ricerca è il fornire implementazioni come prova di principio, 
% simulazioni e risultati sperimentali di algoritmi di \ac{MLQ} selezionati a partire da 
% insiemi di dati di grandezza da piccola a media, osservando come cambiano i requisiti ed 
% i risultati passando da un caso all'altro. A questo scopo, l'algoritmo \acf{KNN}, uno tra i 
% più semplici algoritmi di \ac{ML}, è stato scelto come buon esempio minimale per 
% l'implementazione, la simulazione e la sperimentazione quantistica. 
% Questo è un passo necessario nel tentativo di far passare il \ac{MLQ} da un ambito di 
% ricerca quasi puramente teorico ad un campo più applicativo come è il \ac{ML} classico. 

% Il \ac{ML} classico è un ambito molto concreto poiché può essere direttamente testato, 
% verificato e implementato su qualsiasi computer classico commerciale. 
% Fino ad ora, il \ac{MLQ} è stato per la gran parte di natura teorica, dato che le risorse 
% computazionali richieste sono solo parzialmente sviluppate. Per produrre soluzioni affidabili, 
% gli algoritmi di \ac{MLQ} spesso richiedono un numero enorme di qubit con correzione 
% degli errori e qualche sistema di memorizzazione di dati quantistici, per esempio la 
% \ac{QRAM} proposta da Giovannetti, Lloyd e Maccone. % Fonti
% Io ho usato quella proposta da Petruccione, devo vedere se inserire il riferimento al posto 
% di questo o introdurlo dopo nel discorso

% Al giorno d'oggi il massimo numero di qubit p.e. a superconduzione, da quanto 
% viene riferito, usati per il calcolo sono quattordici; 
% si intende quelli a disposizione pubblica, poi il massimo al mondo bisogna verificarlo
% il D-Wave II quantum annealing device fornisce 1152 qubit ma può risolvere solo una ristretta 
% classe di problemi, e una \ac{QRAM} dedicata non è stata ancora sviluppata. % Fonti
% Verificare che sia ancora così
% Inoltre, la correzione degli errori dei qubit è un campo di ricerca ancora attivo 
% e sebbene ci siano stati notevoli miglioramenti, % mettere nota a margine con dati del miglioramento
% la maggior parte dei \ac{CQ} preliminari descritti hanno a che fare con qubit senza correzione 
% d'errore, con vite brevi e sono, quindi, inadatti per grandi implementazioni di \ac{MLQ}. 

\section{Obiettivo di ricerca} % Sostituire con altra formula

% Alla luce della natura teorica dell'attuale ricerca nel \ac{MLQ} 
% e del piccolo numero di realizzazioni sperimentali, questa ricerca si dedicherà 
% alla seguente domanda: 
Prendendo come punto di partenza gli algoritmi proposti nell'ambito di ricerca del \ac{QML} 
questa tesi proverà a rispondere alla seguente domanda:
\begin{quote}
    è possibile implementare su un computer quantistico un algoritmo k-nearest neighbours 
    multiclasse, in modo da migliorare le prestazioni ed il numero di problemi risolvibili?
    % quali sono le prestazioni di un algoritmo k-nearest neighbours di quantum 
    % machine learning implementato su hadware quantistico di piccola e media dimensione 
    % attualmente disponibili e come scala all'aumentare delle risorse?
    % quali sono le opportunità di implementazione pratica e l'efficienza di un algoritmo 
    % k-nearest neighbours di machine learning quantistico su computer quantistici di piccola 
    % e media scala attualmente disponibili?
\end{quote}
I capitoli seguenti introdurranno le basi teoriche necessarie e gli strumenti usati 
per rispondere a questa domanda. 

%*****************************************
%*****************************************
%*****************************************
%*****************************************
%*****************************************
