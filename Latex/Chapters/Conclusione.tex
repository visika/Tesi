%************************************************
\chapter{Conclusione}\label{ch:conclusione}
%************************************************

Partendo da un semplice classificatore \ac{KNN} classico 
si è riusciti ad implementarne la versione quantistica proposta 
da Schuld et al. \cite{schuld}. Dopo di che si è tentato di 
estenderne le funzioni, con l'intento principale di rendere il 
classificatore capace di distinguere la classe corretta tra 
tutte quelle dell'insieme dati. È stata usata la tecnica di 
costruzione quantum database arbitrari \ac{FF-QRAM} proposta da 
Petruccione et al. \cite{petruccione}.L'algoritmo ha avuto un 
discreto successo e fornisce una distribuzione di previsioni 
che conferma le speranze in una applicazione anche a livello 
commerciale di tecniche di machine learning quantistico. 

Per questioni di limitatezza di tempo, non è stato possibile 
sottoporre ad analisi anche altri data set o implementare tecniche 
più raffinate di preprocessing dei dati. Il lettore interessato 
ad approfondire la materia in maniera più sistematica si può 
affidare a lavori analoghi facilmente reperibili 
online,\footnote{Si veda 
\url{https://github.com/carstenblank/dc-qiskit-qml}.} o ai tanti 
articoli sull'argomento che escono ogni anno in libera consultazione. 

\sectionmark{Commento}

Lavorare a questa tesi mi ha permesso di esplorare le terre, 
a me precedentemente sconosciute, del machine learning e del 
quantum computing. Nonostante il livello di approfondimento 
delle materie raggiunto in questi mesi di lavoro sia ridicolo, 
mi ritengo soddisfatto di essere riuscito a mettere a frutto 
le mie pregresse conoscenze di informatica per partecipare a 
qualcosa di estremamente innovativo. La critica che faccio a 
me stesso è di avere iniziato a scrivere ben troppo tardi il 
testo della tesi; se dovessi ripetere questa esperienza, 
cercherei di fissare bene fin dall'inizio gli obiettivi di 
ricerca e tenere traccia dei progressi in maniera più precisa, 
aggiornando in corso d'opera il documento finale. 