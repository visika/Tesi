%************************************************
\chapter{Conclusione}\label{ch:conclusione}
%************************************************

Partendo da un semplice classificatore \ac{KNN} classico 
si è riusciti ad implementarne la versione quantistica proposta 
da Schuld et al. \cite{schuld}. Dopo di che si è tentato di 
estenderne le funzioni, con l'intento principale di rendere il 
classificatore capace di distinguere la classe corretta tra 
tutte quelle dell'insieme dati e non solo per due alla volta. 
È stata usata la tecnica di 
costruzione di quantum database arbitrari \ac{FF-QRAM} proposta da 
Petruccione et al. \cite{petruccione}. L'algoritmo ha avuto un 
discreto successo e fornisce una distribuzione di previsioni 
in linea con le aspettative, sebbene le esecuzioni su hardware 
reale siano ancora significativamente in balia dei fenomeni di 
rumore. L'avvento di computer quantistici con un numero maggiore 
di qubit può aprire la strada ad approcci concreti di analisi 
dati con data set di dimensioni e complessità considerevoli, 
lasciando intravedere un'opportunità di applicazione 
commerciale nei confronti dei big data. 

Per questioni di limitatezza di tempo, non è stato possibile 
sottoporre ad analisi anche altri data set o implementare tecniche 
più raffinate di preprocessing dei dati. Il lettore interessato 
ad approfondire la materia in maniera più sistematica si può 
affidare a lavori analoghi facilmente reperibili 
online\footnote{Si veda ad esempio 
\url{https://github.com/carstenblank/dc-qiskit-qml}.}, alla comunità 
di Qiskit su \url{https://quantumcomputing.stackexchange.com/} o ai tanti 
articoli sull'argomento che escono ogni anno in libera consultazione. 

Un'interessante proseguimento di questo progetto potrebbe essere 
la messa a disposizione degli algoritmi qui studiati per chi volesse 
usarli per analizzare data set personalizzati. Questo potrebbe 
essere realizzato attraverso un'ulteriore ottimizzazione e 
documentazione del codice pubblicato su GitHub, oppure con la 
creazione di un programma completo apposito, o ancora con la 
costruzione di un'interfaccia web in cui caricare i propri file 
(CSV per esempio) e lasciare che il server ottimizzi i dati, crei 
il circuito quantistico e mandi l'ordine di esecuzione correlato. 

\section{Commento}

Lavorare a questa tesi mi ha permesso di esplorare le terre, 
del machine learning e del quantum computing, due materie 
fortemente moderne ed in continua evoluzione. 
Nonostante l'approfondimento delle materie compiuto in 
questi mesi di lavoro sia tutt'altro che completo, 
mi ritengo soddisfatto di essere riuscito a mettere a frutto 
le mie pregresse conoscenze di informatica per partecipare a 
qualcosa di estremamente innovativo. La critica che faccio a 
me stesso è di avere iniziato a scrivere ben troppo tardi il 
testo della tesi; se dovessi ripetere questa esperienza, 
cercherei di fissare bene fin dall'inizio gli obiettivi di 
ricerca e tenere traccia dei progressi in maniera più assidua, 
aggiornando in corso d'opera il documento finale. 