%************************************************
\chapter{Strumenti}\label{ch:strumenti}
%************************************************

L'intera ricerca per questa tesi è stata effettuata con carta e penna ed un 
computer con il sistema operativo GNU/Linux Ubuntu. Il linguaggio di 
programmazione Python, con la sua sintassi molto intuitiva e pacchetti estesi 
per il calcolo scientifico e il disegno di grafici, è stato usato per i calcoli 
e per la maggior parte dei grafici in questa tesi. 
Per l'implementazione di un algoritmo \ac{KNN}, ci sono principalmente due 
strade: eseguirlo simulando un \ac{CQ} o eseguirlo proprio su un \ac{CQ} reale. 
Gli strumenti necessari per entrambi gli approcci saranno descritti nelle 
sezioni seguenti. 

\section{Qiskit}

Qiskit è un'interfaccia di programmazione che permette di scrivere 
circuiti quantistici e simularne l'esecuzione sul proprio computer 
o inviare un ordine di esecuzione a un vero computer quantistico tramite 
l'interfaccia offerta dall'IBM Quantum Experience.  

\section{IBM QX}

L'IBM Quantum Experience è un servizio offerto gratuitamente che permette 
a chiunque di avere a che fare con un computer quantistico. Sono presenti 
risorse didattiche per imparare a scrivere il primo circuito, strumenti 
di comunità come una piattaforma di domande e risposte e soprattutto 
un sistema per creare i propri algoritmi. 